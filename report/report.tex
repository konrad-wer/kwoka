\documentclass[a4paper]{article}

\usepackage[T1]{fontenc}
\usepackage[utf8]{inputenc}
\usepackage{float}
\usepackage{caption}
\usepackage[table]{xcolor}
\usepackage[english]{babel}
\usepackage{amsmath, amsfonts}
\usepackage{indentfirst}
\usepackage{graphicx}
\usepackage{hyperref}
\usepackage{mathtools, array, mathpartir, amssymb, stmaryrd, cancel, cleveref, fancyvrb}
\usepackage[nofoot,hdivide={2cm,*,2cm},vdivide={2cm,*,2cm}]{geometry}
\frenchspacing
\pagenumbering{gobble}

\newenvironment{grammar}[2]
 {\begin{tabular}{@{\qquad}>{$}l<{$}@{\qquad}l@{}}
  \multicolumn{1}{@{}l@{}}{$#1$}&\multicolumn{1}{l@{}}{\hspace{-2em}#2}\\}
 {\end{tabular}}

\author{Konrad Werbliński - 291878}
\title{Programming languages - final project}
\date{\today}

\begin{document}
\maketitle

\section{Introduction}
Algebraic effects have received lot of recognition in
the functional programming community in recent years.
They are a novel approach for handling side effects.
Effectful computations were always considered problematic by
programming language researchers because they make programs harder to
formally describe and reason about. Moreover, restricting side effects helps to
write more maintainable and easier to test programs.
With the growing popularity of the Haskell language, monads became preferred way of
expressing side effects in a pure way. However, they do not compose very well.
Algebraic effects can be viewed as resumable exceptions, providing more composable
approach for handling side effects than monads.

We introduce Kwoka - stripped down and simplified version of the Koka language \cite{Koka}.
In this report we describe our work on implementation of the
type inference \cite{leijen2017type} and abstract machine \cite{HandleWithCare} for algebraic effects.

In the next section we specify our variant of a language. We describe the
details of the implemented type system and abstract machine.
Then, we give an overview of the implementation and instructions for building and running Kwoka.
Finally we discuss further work on this project.

\section{Kwoka language description}
Kwoka syntax is based on the syntax of the original Koka language and Rust.
It is designed to resemble popular imperative languages, thus providing familiarity
for programmers that don't know the functional paradigm.
To articulate the familiarity even more and simplify working with effects, multi-argument
functions are defined in the uncurried form.
\begin{verbatim}
fn fib(n)
{
  if n == 0 || n == 1 then
    n
  else
    fib(n - 1) + fib(n - 2)
}
\end{verbatim}
Kwoka supports higher order functions and let-polymorphism
(generalization occurs for let expressions, but also for top level function definitions).
\begin{verbatim}
fn compose(f, g)
{
  fn (x) => f(g(x))
}
\end{verbatim}
The language also features product and list types.
\begin{verbatim}
fn fst(p)
{
  let (x, y) = p in x
}

fn map(f, xs)
{
  case xs of
    [] => []
    x : xs => f(xs) : map(f, xs)
}
\end{verbatim}
\subsection*{Algebraic effects}
The key feature of the Kwoka language are algebraic effects.
Effects are defined as the top level definitions. In the definition we specify possible
operations for that effect. For simplicity types of operations are not polymorphic.
For handling the effects we use \verb+handle+ expression.
Inside the \verb+handle+ expression programmer has to provide ways of handling all of the
effect's operations and handling of the return value of the effectful computation.
Popular example of usage of algebraic effects is using them to
express exceptions. Thus, they don't have to be part of the language anymore.

\begin{verbatim}
effect Exc
{
  Raise(String)
}

fn saveDiv(n, m)
{
  if m == 0 then
    Raise("Division by zero")
  else
    n / m
}

fn main()
{
  let y = ReadLnInt() in
  handle<Exc>(saveDiv(400, y))
  {
    return(x) => PrintInt(x),
    Raise(s) => PutStrLn("Division by zero")
  }
}
\end{verbatim}
Algebraic effects can be also used for simulating computations with nondeterminism.
In the example below we define effect that represents the coin toss. Then, in the handler of that effect
we use multiple resumptions to generate all sublists of the list.
\begin{verbatim}
effect Toss
{
  Coin() :: Bool
}

fn sublist(xs)
{
  case xs of
    [] => []
    x : xs =>
      if Coin() then
        x : sublist(xs)
      else
        sublist(xs)
}

fn main()
{
  let res = handle<Toss>(sublist([1,2,3,4,5]))
  {
    return(x) => [x],
    Coin() => resume(True) @ resume(False)
  } in

  iter (fn (xs) => let () = iter(fn (x) => PrintInt(x) , xs) in PutStrLn(""), res)
}
\end{verbatim}
Kwoka provides default definition and handler for the IO effect,
which consists of following operations:
\verb+GetLine+, \verb+ReadLnInt+, \verb+PutStrLn+, \verb+PrintInt+.

\subsection*{Definition scope}
Top level functions are visible to themselves and functions below them.
Effect definitions are visible in the entire source file.

\section{Type Inference}
\textbf{Types:}

\begin{grammar}{\tau\Coloneqq}{}
  \verb+Bool+ \mid \verb+Int+ \mid \verb+String+               & simple types\\
  \mid  (\tau_1, \tau_2, \dots \tau_n)                         & product\\
  \mid [\tau]                                                  & lists\\
  \mid \tau_1 \rightarrow \langle \epsilon \rangle \tau_2      & arrows\\
  \mid \alpha                                                  & type variable
\end{grammar}\bigskip\\

\textbf{Type schemes:}

\begin{grammar}{\sigma\Coloneqq}{}
  \forall \overline\alpha.\, \tau
\end{grammar}\bigskip\\

\textbf{Effect types}

\begin{grammar}{\epsilon\Coloneqq}{}
  \langle l \mid \epsilon \rangle              & label\\
  \mid \mu                                     & effect variable\\
  \mid \langle \rangle                         & empty effect row
\end{grammar}\bigskip\\

Kwoka uses the Hindley-Milner type system \cite{Hindley, Milner} extended for algebraic effects.
We follow Daan Leijen's \cite{leijen2017type} approach of using extensible rows with
scoped labels for effect tracking. The concept of extensible rows was originally proposed for implementing
extensible records with scoped labels \cite{leijen2005extensible}, however they fit very
well for tracking the effects of computations.

\begin{mathpar}
  \inferrule{\Gamma(x) = \sigma}
            {\Gamma \vdash  x : \sigma \mid \epsilon} \and
  \inferrule{\Gamma \vdash  e_1 : \sigma \mid \epsilon \\
             \Gamma, x : \sigma \vdash  e_2 : \tau \mid \epsilon}
            {\Gamma \vdash \text{let } x = e_1 \text{ in } e_2 : \tau \mid \epsilon} \and
  \inferrule{\Gamma \vdash e : \tau \mid \langle\rangle \\
             \overline{\alpha} \notin \mathrm{ftv}(\Gamma)}
            {\Gamma \vdash e : \forall \overline{\alpha}.\tau \mid \epsilon} \and
  \inferrule{\Gamma \vdash e : \forall \overline{\alpha}.\tau \mid \epsilon}
            {\Gamma \vdash e : \tau [\overline{\alpha} \mapsto \overline{\tau}]\mid \epsilon} \and
  \inferrule{\Gamma, x : \tau_1 \vdash e : \tau_2 \mid \epsilon'}
            {\Gamma \vdash \lambda x. e : \tau_1 \rightarrow \epsilon' \tau_2 \mid \epsilon} \and
  \inferrule{\Gamma \vdash  e_1 : \tau_1 \rightarrow \epsilon \, \tau_2 \mid \epsilon \\
             \Gamma \vdash  e_2 : \tau_1 \mid \epsilon}
            {\Gamma \vdash e_1(e_2) : \tau_2 \mid \epsilon}
\end{mathpar}
\vspace{1cm}
\begin{mathpar}
  \inferrule{\Gamma \vdash  e: \tau \mid \langle l \mid \epsilon \rangle \\
             \Sigma(l) = \{op_1, \dots, op_n\} \\
             \Gamma, x : \tau \vdash e_r : \tau_r \mid \epsilon \\
             \Gamma \vdash op_i : \tau_i \rightarrow \langle l \rangle \tau_i' \mid \langle \rangle \\
             \Gamma, \mathit{resume} : \tau_i' \rightarrow \epsilon \, \tau_r, x_i : \tau_i \vdash e_i : \tau_r \mid \epsilon}
            {\Gamma \vdash \textbf{handle$\langle l\rangle$}(e)\{op_1(x_1) \rightarrow e_1; \, \dots; \, op_n(x_n) \rightarrow e_n; \,
             \mathbf{return}(x) \rightarrow e_r \} : \tau_r \mid \epsilon}
\end{mathpar}

\vspace{1cm}

We use the unification algorithm described in the Leijen's
article \cite{leijen2005extensible}.
We omit the unification rules for product and list types, since they are straightforward.
\begin{center}
  $c \Coloneqq$ \verb+Bool+ $\mid$ \verb+Int+ $\mid$ \verb+String+
\end{center}
\begin{mathpar}
  \inferrule{}
            {c \sim c: []} \and
  \inferrule{}
            {\alpha \sim \alpha: []} \and
  \inferrule{\alpha \notin \mathrm{ftv}(\tau)}
            {\alpha \sim \tau: [\alpha \mapsto \tau]} \and
  \inferrule{\alpha \notin \mathrm{ftv}(\tau)}
            {\tau \sim \alpha : [\alpha \mapsto \tau]} \and
  \inferrule{\tau_1 \sim \tau_1' : \Theta_1 \\
             \Theta_1\epsilon \sim \Theta_1\epsilon' : \Theta_2 \\
             (\Theta_2 \circ \Theta_1)\tau_2 \sim (\Theta_2 \circ \Theta_1)\tau_2' : \Theta_3}
            {\tau_1 \rightarrow \langle \epsilon \rangle \tau_2 \sim
             \tau_1' \rightarrow \langle \epsilon' \rangle \tau_2': \Theta_3 \circ \Theta_2 \circ \Theta_1} \and
  \inferrule{}
             {\langle\rangle \sim \langle\rangle: []} \and
  \inferrule{}
             {\mu \sim \mu: []} \and
  \inferrule{\mu \notin \mathrm{ftv}(\epsilon)}
             {\mu \sim \epsilon: [\mu \mapsto \epsilon]} \and
  \inferrule{\mu \notin \mathrm{ftv}(\epsilon)}
             {\epsilon \sim \mu : [\mu \mapsto \epsilon]} \and
  \inferrule{\epsilon_2 \simeq \langle l \mid \epsilon_2' \rangle : \Theta_1 \\
            \mathrm{tail}(\epsilon_1) \notin \mathrm{dom}(\Theta_1) \\
            \Theta_1 \epsilon_1 \sim \Theta_1 \epsilon_2' : \Theta_2}
            {\langle l \mid \epsilon_1 \rangle \sim \epsilon_2 : \Theta_2 \circ \Theta_1}
\end{mathpar}

\begin{mathpar}
  \inferrule{}
            {\langle l \mid \epsilon \rangle \simeq \langle l \mid \epsilon \rangle: []} \and
  \inferrule{l \neq l' \\
             \epsilon \simeq \langle l \mid \epsilon' \rangle: \Theta}
            {\langle l' \mid \epsilon \rangle \simeq \langle l \mid  l' \mid \epsilon' \rangle: \Theta} \and
  \inferrule{\mathrm{fresh}(\mu')}
            {\mu \simeq \langle l \mid \mu' \rangle : [\mu \mapsto \langle l \mid \mu' \rangle]}
\end{mathpar}

\section{Abstract machine}
We implemented slightly modified version of the abstract machine introduced by
Dariusz Biernacki, Maciej Piróg, Piotr Polesiuk and Filip Sieczkowski \cite{HandleWithCare}.
$$e \Rightarrow \langle e, \{\}, \bullet, \bullet \rangle_{eval} $$

$$\langle x, \rho, \kappa, \pi \rangle_\mathit{eval} \Rightarrow
  \langle \kappa, v, \pi \rangle_\mathit{stack} \; \text{where} \; v = \rho(x)$$
$$\langle \lambda x.e, \rho, \kappa, \pi \rangle_\mathit{eval} \Rightarrow
  \langle \kappa, \lambda^\rho x.e, \pi \rangle_\mathit{stack}$$
$$\langle (), \rho, \kappa, \pi \rangle_\mathit{eval} \Rightarrow
  \langle \kappa, (), \pi \rangle_\mathit{stack}$$
$$\langle (e_1, e_2, \dots, e_n), \rho, \kappa, \pi \rangle_\mathit{eval} \Rightarrow
   \langle e_1, \rho, \{() (e_2, \dots, e_n) \}^\rho : \kappa, \pi \rangle_\mathit{eval}$$
$$\langle \mathit{op}_l, \rho, \kappa, \pi \rangle_\mathit{eval} \Rightarrow \langle \kappa, \mathit{op}_l, \pi \rangle_\mathit{stack}$$
$$\langle e_1 e_2, \rho, \kappa, \pi \rangle_\mathit{eval} \Rightarrow
   \langle e_1, \rho, e_2^\rho : \kappa, \pi \rangle_\mathit{eval}$$
$$\langle \mathit{primitive}, \rho, \kappa, \pi \rangle_\mathit{eval} \Rightarrow
  \langle \kappa, \mathit{primitive}, \pi \rangle_\mathit{stack}$$
$$\langle \textbf{handle$\langle l\rangle$}(e)\{\mathit{hr}\}, \rho, \kappa, \pi \rangle_\mathit{eval} \Rightarrow
  \langle e, \rho, \bullet, (\mathit{hr}_l^\rho, \kappa) : \pi \rangle_\mathit{eval}$$

$$\langle \bullet, v, \pi \rangle_\mathit{stack} \Rightarrow \langle \pi, v \rangle_\mathit{mstack}$$
$$\langle e^\rho : \kappa, v, \pi \rangle_\mathit{stack} \Rightarrow
  \langle e, \rho, v : \kappa, \pi \rangle_\mathit{eval}$$
$$\langle \{(v_1, \dots, v_{n-1})()\}^\rho : \kappa, v_n, \pi \rangle_\mathit{stack} \Rightarrow
  \langle \kappa, (v_1, \dots, v_{n-1}, v_n), \pi \rangle_\mathit{stack}$$
$$\langle \{(v_1, \dots, v_{k-1})(e_{k+1}, e_{k+2}, \dots, e_n)\}^\rho : \kappa, v_k, \pi \rangle_\mathit{stack} \Rightarrow
  \langle e_{k+1}, \rho, \{(v_1, \dots, v_{k-1}, v_k) (e_{k+2}, \dots, e_n) \}^\rho : \kappa, \pi \rangle_\mathit{eval}$$
$$\langle \lambda^\rho x.e : \kappa, v, \pi \rangle_\mathit{stack} \Rightarrow
  \langle e, \rho\{x \mapsto v\}, \kappa, \pi \rangle_\mathit{eval}$$
$$\langle \mathit{primitive} : \kappa, v, \pi \rangle_\mathit{stack} \Rightarrow
  \langle \kappa, v', \pi \rangle_\mathit{stack}  \; \text{where} \; v' = \llbracket \mathit{primitive} \; v \rrbracket$$
$$\langle \mathit{op}_l : \kappa, v, \pi \rangle_\mathit{stack} \Rightarrow
  \langle \mathit{op}_l, \kappa, \pi, v, \bullet \rangle_\mathit{op}$$
$$\langle \theta : \kappa, v, \pi \rangle_\mathit{stack} \Rightarrow
  \langle \theta , \kappa, \pi, v \rangle_\mathit{res}$$

$$\langle \mathit{op}_l, \kappa, (\mathit{\{h;d\}}_l^\rho, \kappa') : \pi, v, \theta \rangle_\mathit{op} \Rightarrow
   \langle e, \rho\{x \mapsto v\}\{r \mapsto (\mathit{\{h;d\}}_l^\rho, \kappa) : \theta\}, \kappa', \pi \rangle_\mathit{eval}
   \; \text{where} \; op \, x,\, r.e \in h$$
$$\langle \mathit{op}_l, \kappa, (\mathit{hr}_{l'}^\rho, \kappa') : \pi, v, \theta \rangle_\mathit{op} \Rightarrow
  \langle \mathit{op}_l, \kappa', \pi, v, (\mathit{hr}_{l'}^\rho, \kappa) : \theta \rangle_\mathit{op}
   \; \text{if} \; l \neq l'$$

$$\langle \bullet, \kappa, \pi, v \rangle_\mathit{res} \Rightarrow
  \langle \kappa, v, \pi \rangle_\mathit{stack}$$
$$\langle (\mu, \kappa') : \theta, \kappa, \pi, v \rangle_\mathit{res} \Rightarrow
  \langle \theta, \kappa', (\mu, \kappa) : \pi, v \rangle_\mathit{res}$$

$$ \langle (\{h; \mathbf{return}(x).\,e\}_l^\rho, \kappa) : \pi, v \rangle_\mathit{mstack} \Rightarrow
   \langle e, \rho\{x \mapsto v\}, \kappa, \pi \rangle_\mathit{eval}$$
$$ \langle \bullet, v \rangle_\mathit{mstack} \Rightarrow v$$

\section{Implementation}
Kwoka is implemented in Haskell, as we believe this language has cleaner syntax and
more elegant way of handling effectful computations than other mainstream functional languages.
\subsection*{Source code structure}

\subsubsection*{AST}
In the AST module we define types to describe syntax of the expressions,
effect definitions and types. Moreover, we provide pretty-printing algorithm
by making all of the above mentioned data types instances of the \verb+Show+ type class.
\begin{itemize}
\item Expressions:
\begin{verbatim}
data Expr p
  = EVar      p Var
  | EBool     p Bool
  | EInt      p Integer
  | EString   p String
  | ENil      p
  | EUnOp     p UnOp (Expr p)
  | EBinOp    p BinOp (Expr p) (Expr p)
  | ELambda   p [Var] (Expr p)
  | EApp      p (Expr p) (Expr p)
  | EIf       p (Expr p) (Expr p) (Expr p)
  | ELet      p Var (Expr p) (Expr p)
  | EOp       p Var (Expr p)
  | EHandle   p Var (Expr p) [Clause p]
  | ETuple    p [Expr p]
  | ELetTuple p [Var] (Expr p) (Expr p)
  | ECase     p (Expr p) (Expr p) (Var, Var) (Expr p)
\end{verbatim}
\item Types:
\begin{verbatim}
data Type
  = TVar TypeVar
  | TBool
  | TInt
  | TString
  | TList Type
  | TProduct [Type]
  | TArrow Type EffectRow Type
  deriving Eq
\end{verbatim}
\item Type schemes:
\begin{verbatim}
data TypeScheme = TypeScheme [TypeVar] Type
\end{verbatim}
\item Effect rows:
\begin{verbatim}
data EffectRow
  = EffLabel Var EffectRow
  | EffEmpty
  | EffVar TypeVar
  deriving Eq
\end{verbatim}
\end{itemize}
\subsubsection*{Parser}
For parsing we use the megaparsec library,
which is an extended and improved version of the popular parsec library for monadic parsing.

\subsubsection*{Preliminary}
After the parsing phase we move to preliminary phase, where type and effect environments
are built from the effect definitions. In this module we also check the well-formedness of the definitions
and define the types of the builtin operators.

\subsubsection*{Type inference}
In the type inference module we implement the Hindley-Milner type inference algorithm extended
for handling algebraic effects. We use monad transformers to compose \verb+State+ and \verb+Either+
monads, used for fresh variable names generating and type error signaling. Aside from that, in this
module we define type for representing type errors and we make it an instance of the \verb+Show+
type class for converting them into the human readable form.

\begin{itemize}
  \item \verb+infer+ - main function of the inference algorithm, infers the types of expressions.
  \item \verb+check+ - helper function that infers expression type and then unifies it with a given type.
  \item \verb+unify+ - unification of types.
  \item \verb+unifyRow+ - unification of effect rows.
\end{itemize}

\subsubsection*{MachineAST}
In the MachineAST module we define types that represent expressions in the abstract machine language.
We define primitive operations which represent unary and binary operators and IO operations
(used for implementation of the default IO effect handler, which is written in the abstract machine code).
In this module we also define data structures necessary for the machine to work, such as stack frames and values.

\begin{itemize}
\item Machine expressions:
\begin{verbatim}
data MExpr
  = MVar     MVar
  | MInt     Integer
  | MBool    Bool
  | MString  String
  | MNil
  | MTuple   [MExpr]
  | MPrim    MPrim MExpr
  | MLambda  [MVar] MExpr
  | MApp     MExpr MExpr
  | MIf      MExpr MExpr MExpr
  | MOp MVar MVar MExpr
  | MCase    MExpr MExpr (MVar, MVar) MExpr
  | MHandle  MVar MExpr (Map.Map MVar MClause)
  deriving (Show, Eq)
\end{verbatim}
\item Primitive operations:
\begin{verbatim}
data MPrim
  = Not | Neg | Mult | Div | Mod | Add | Sub
  | Concat | Cons | Append | Equal | NotEqual
  | LessEqual | GreaterEqual | Less | Greater
  | And | Or | Print | GetLine | ReadLnInt deriving (Show, Eq)
\end{verbatim}
\item Values:
\begin{verbatim}
data MValue
  = VInt Integer
  | VBool Bool
  | VString String
  | VList [MValue]
  | VTuple [MValue]
  | VClosure RuntimeEnv [MVar] MExpr
  | VReifiedMetaStack MetaStack
\end{verbatim}
\item Stack frames:
\begin{verbatim}
data MFrame
  = FArg RuntimeEnv MExpr
  | FClosure RuntimeEnv [MVar] MExpr
  | FOp MVar MVar
  | FPrim MPrim
  | FTuple RuntimeEnv [MValue] [MExpr]
  | FIf RuntimeEnv MExpr MExpr
  | FCase RuntimeEnv MExpr (MVar, MVar) MExpr
  | FReifiedMetaStack MetaStack
  deriving (Show, Eq)
\end{verbatim}
\item Meta-stack frames:
\begin{verbatim}
data MMetaFrame
  = MMetaFrame (MVar, Map.Map MVar MClause, RuntimeEnv) Stack
  deriving (Show, Eq)
\end{verbatim}
\end{itemize}

\subsubsection*{Translate}
In this module we implement translation algorithm between abstract syntax and machine code.
During this translation we also perform basic constant folding.

\begin{itemize}
  \item \verb+translate+ - translates expressions to machine expressions.
  \item \verb+binOpToMPrim+ - translates binary operators to primitive operations.
  \item \verb+ioHandler+ - implementation of the default IO handler,
              which is placed over the main function.
\end{itemize}

\subsubsection*{Machine}
In this module we define five functions which correspond to five configurations of the abstract machine.
We use the IO monad to implement primitive operations for input-output.
\begin{itemize}
  \item \verb+eval :: MExpr -> RuntimeEnv -> Stack -> MetaStack -> IO MValue+
  \item \verb+stack :: Stack -> MValue -> MetaStack -> IO MValue+
  \item \verb+op :: (MVar, MVar) -> Stack -> MetaStack -> MValue -> MetaStack -> IO MValue+
  \item \verb+resume :: MetaStack -> Stack -> MetaStack -> MValue -> IO MValue+
  \item \verb+mstack :: MetaStack -> MValue -> IO MValue+
\end{itemize}

\subsubsection*{Main}

In the main module we bind all of the implementation parts together, handle reading the command line arguments and
handle loading the source file.

\subsubsection*{Tests}
In the \verb+test+ folder we provide over 85 unit tests for the type system.

\section{Installing and running Kwoka}
Kwoka requires {\href{https://docs.haskellstack.org/en/stable/README/}{\color{blue}stack}} for building.
\begin{itemize}
    \item Building: \verb+stack build+.
    \item Running: \verb+stack run -- yourSourceFile.kwoka [-debug]+.
    \item Adding \verb+-debug+ flag pretty-prints typed typed program to a file.
    \item Running unit tests: \verb+stack test+.
    \item To install kwoka on your machine: \verb+stack install+,
          then you can run kwoka in your terminal by running \verb+kwoka-exe+.
\end{itemize}
In the \verb+example+ folder we provide a handful of examples of the Kwoka language.

\section{Further work}
\begin{itemize}
  \item We plan to implement the type directed selective CPS translation \cite{leijen2017type}
        as a next step towards efficient programming language implementation.
  \item Useful feature would be inclusion of effect operations with polymorphic types.
  \item We plan to refactor unit tests to use popular testing framework for Haskell: HUnit.
\end{itemize}


\bibliographystyle{unsrt}
\bibliography{biblo.bib}

\end{document}