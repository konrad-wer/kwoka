\documentclass[a4paper]{article}

\usepackage[T1]{fontenc}
\usepackage[utf8]{inputenc}
\usepackage{float}
\usepackage{caption}
\usepackage[table]{xcolor}
\usepackage[english]{babel}
\usepackage{amsmath, amsfonts}
\usepackage{indentfirst}
\usepackage{graphicx}
\usepackage{hyperref}
\usepackage{mathtools, array, mathpartir, amssymb, stmaryrd, cancel, cleveref, fancyvrb}
\usepackage[nofoot,hdivide={2cm,*,2cm},vdivide={2cm,*,2cm}]{geometry}
\frenchspacing

\author{Konrad Werbliński - 291878}
\title{Programming languages - final project}
\date{\today}

\begin{document}
\maketitle

\section{Introduction}
Algebraic effects have received lot of recognition in
the functional programming community in recent years.
They are a novel approach for handling side effects.
Effectful computations were always considered problematic by
programming language researchers because they make programs harder to
formally describe and reason about. Moreover, restricting side effects helps to
write more maintainable and easier to test programs.
With the growing popularity of the Haskell language, monads became preferred way of
expressing side effects in a pure way. However, they do not compose very well.
Algebraic effects can be viewed as resumable exceptions, providing more composable
approach for handling side effects than monads.

We introduce Kwoka - stripped down and simplified version of the Koka language \cite{Koka}.
In this report we describe our work on implementation of the
type inference for algebraic effects introduced
in the article by Daan Leijen \cite{leijen2017type}.
We follow his approach of using extensible rows with
scoped labels for tracking effects of computations.

In this report we specify our variant of a language. We describe the
details of the implemented type system.
Then, we give an overview of the implementation and instructions for building and running Kwoka.
Finally we discuss further work on this project.

\section{Kwoka language description}
Kwoka syntax is based on the syntax of the original Koka language and Rust.
It is designed to resemble popular imperative languages, thus providing familiarity
for programmers that don't know the functional paradigm.
To articulate the familiarity even more and simplify working with effects, multi-argument
functions are defined in the uncurried form.
\begin{verbatim}
fn fib(n)
{
  if n == 0 || n == 1 then
    n
  else
    fib(n - 1) + fib(n - 2)
}
\end{verbatim}
Kwoka supports higher order functions and let-polymorphism
(generalization occurs for let expressions, but also for top level function definitions).
\begin{verbatim}
fn compose(f, g)
{
  \x => f(g(x))
}
\end{verbatim}
\subsection*{Algebraic effects}
The key feature of the Kwoka language are algebraic effects.
Effects are defined as the top level definitions. In the definition we specify possible
actions for that effect. For simplicity types of actions are not polymorphic.
For handling the effects we the use \verb+handle+ expression.
Inside the \verb+handle+ expression programmer has to provide ways of handling all of the
effect's actions and handling of the return value of the effectful computation.
\begin{verbatim}
effect Hello
{
  Hello() :: String
}

fn makeGreeting(name)
{
  Hello() ^ " " ^ name
}

fn main()
{
  handle<Hello>(makeGreeting("General Kenobi."))
  {
    return(x) => (),
    Hello() => resume("Hello there!")
  }
}
\end{verbatim}

Popular example of usage of the algebraic effects is using them to
express exceptions. Thus, they don't have to be part of the language anymore.

\begin{verbatim}
effect Exc
{
  Raise(String)
}

fn saveDiv(n, m)
{
  if m == 0 then
    Raise("Division by zero")
  else
    n / m
}

fn main()
{
  handle<Exc>(saveDiv(4, 5))
  {
    return(x) => x,
    Raise(s) => 0
  }
}
\end{verbatim}

\subsection*{Definition scope}
Top level functions are visible to themselves and functions below them.
Effect definitions are visible in the entire source file.

\section{Type Inference}
Kwoka uses the Hindley-Milner type system \cite{Hindley, Milner} extended for handling
inference of the effects. Effects are represented as extensible rows with
scoped labels. This concept of extensible rows was originally proposed for implementing
extensible records with scoped labels \cite{leijen2005extensible}, however they fit very
well for tracking the effects of computations.

\begin{mathpar}
  \inferrule{\Gamma(x) = \sigma}
            {\Gamma \vdash  x : \sigma \mid \epsilon} \and
  \inferrule{\Gamma \vdash  e_1 : \sigma \mid \epsilon \\
             \Gamma, x : \sigma \vdash  e_2 : \tau \mid \epsilon}
            {\Gamma \vdash \text{let } x = e_1 \text{ in } e_2 : \tau \mid \epsilon} \and
  \inferrule{\Gamma \vdash e : \tau \mid \langle\rangle \\
             \overline{\alpha} \notin \mathrm{ftv}(\Gamma)}
            {\Gamma \vdash e : \forall \overline{\alpha}.\tau \mid \epsilon} \and
  \inferrule{\Gamma \vdash e : \forall \overline{\alpha}.\tau \mid \epsilon}
            {\Gamma \vdash e : \tau [\overline{\alpha} \mapsto \overline{\tau}]\mid \epsilon} \and
  \inferrule{\Gamma, x : \tau_1 \vdash e : \tau_2 \mid \epsilon'}
            {\Gamma \vdash \lambda x. e : \tau_1 \rightarrow \epsilon' \tau_2 \mid \epsilon} \and
  \inferrule{\Gamma \vdash  e_1 : \tau_1 \rightarrow \epsilon \, \tau_2 \mid \epsilon \\
             \Gamma \vdash  e_2 : \tau_1 \mid \epsilon}
            {\Gamma \vdash e_1(e_2) : \tau_2 \mid \epsilon}
\end{mathpar}
\vspace{1cm}
\begin{mathpar}
  \inferrule{\Gamma \vdash  e: \tau \mid \langle l \mid \epsilon \rangle \\
             \Sigma(l) = \{op_1, \dots, op_n\} \\
             \Gamma, x : \tau \vdash e_r : \tau_r \mid \epsilon \\
             \Gamma \vdash op_i : \tau_i \rightarrow \langle l \rangle \tau_i' \mid \langle \rangle \\
             \Gamma, \mathit{resume} : \tau_i' \rightarrow \epsilon \, \tau_r, x_i : \tau_i \vdash e_i : \tau_r \mid \epsilon}
            {\Gamma \vdash \textbf{handle}(e)\{op_1(x_1) \rightarrow e_1; \, \dots; \, op_n(x_n) \rightarrow e_n; \,
             \mathbf{return}(x) \rightarrow e_r \} : \tau_r \mid \epsilon}
\end{mathpar}

\vspace{1cm}

We use the unification algorithm described in the Leijen's
article about extensible records \cite{leijen2005extensible}.

\vspace{2cm}

\begin{center}
  \large TODO Unification rules
\end{center}

\vspace{2cm}

\section{Implementation}
Kwoka is implemented in Haskell, as we believe this language has cleaner syntax and
more elegant way of handling effectful computations than other mainstream functional languages.
\subsection*{Source code structure}

\subsubsection*{AST}
In the AST module we define types to describe syntax of the expressions,
effect definitions and types. Moreover, we provide pretty printing algorithm
by making all of the above mentioned data types instances of the \verb+Show+ type class.

\subsubsection*{Parser}
For parsing we use the megaparsec library,
which is an extended and improved version of the popular parsec library for monadic parsing.

\subsubsection*{Preliminary}
After the parsing phase we move to preliminary phase, where type and effect environments
are built from the effect definitions.

\subsubsection*{Type inference}
In the type inference module we implement the Hindley-Milner type inference algorithm extended
for handling algebraic effects. We use monad transformers to compose \verb+State+ and \verb+Either+
monads, used for fresh variable names generating and type error signaling. Aside from that, in this
module we define type for representing type errors and we make it an instance of the \verb+Show+
type class for converting them into the human readable form.

\subsubsection*{Main}

In the main module we bind the algorithm together, handle reading the command line arguments and
handle loading of the source file.

Upon successful parsing and type checking, Kwoka pretty prints the parsed AST to the standard output.
In the other case, it prints the human readable representation of the encountered error.

\subsubsection*{Tests}
In the \verb+test+ folder we provide over 85 unit tests for the type system.

\section{Installing and running Kwoka}
Kwoka requires {\href{https://docs.haskellstack.org/en/stable/README/}{\color{blue}stack}} for building.
\begin{itemize}
    \item Building: \verb+stack build+
    \item Running: \verb+stack run yourSourceFile.kwoka+
    \item Running unit tests: \verb+stack test+
\end{itemize}
In the \verb+example+ folder we provide a handful of examples of the Kwoka language.

\section{Further work}
\begin{itemize}
  \item We plan to implement the evaluation and the type directed selective CPS translation described in the
  second part of the Daan Leijen's article\cite{leijen2017type}.
  \item Small and simple, but extremely useful extension would be addition of list and product
  types.
  \item Another useful feature would be inclusion of effect actions with polymorphic types.
  \item We plan to refactor unit tests to use popular testing framework for Haskell: HUnit.
\end{itemize}


\bibliographystyle{unsrt}
\bibliography{biblo.bib}

\end{document}