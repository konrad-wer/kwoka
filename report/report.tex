\documentclass[a4paper]{article}

\usepackage[T1]{fontenc}
\usepackage[utf8]{inputenc}
\usepackage{float}
\usepackage{caption}
\usepackage[table]{xcolor}
\usepackage[english]{babel}
\usepackage{amsmath, amsfonts}
\usepackage{indentfirst}
\usepackage{graphicx}
\usepackage{hyperref}
\usepackage[nofoot,hdivide={2cm,*,2cm},vdivide={2cm,*,2cm}]{geometry}
\frenchspacing

\author{Konrad Werbliński - 291878}
\title{Programming languages - final project}
\date{\today}

\begin{document}
\maketitle

\section{Introduction}
  Algebraic effects have received lot of recognition in
  the functional programming community in recent years.
  They can be viewed as resumable exceptions, providing more composable
  approach for handling side effects than monads.
  We introduce Kwoka - stripped down and simplified version of the Koka language \cite{Koka}.
  In this report we describe our work on implementation of the
  type inference for algebraic effects introduced
  in the article by Daan Leijen \cite{leijen2017type}.
  We follow his approach of using extensible effect rows with
  scoped labels for inference of 

  In this report we specify our variant of a language. We describe the
  details of the implemented type system. 
  Then, we give an overview of the implementation and instructions for building and running Kwoka.
  Finally we discuss further work on this project.
\section{Kwoka language description}
Kwoka syntax is based on the syntax of the original Koka language and Rust.
\begin{verbatim}
fn fib(n)
{
  if n == 0 || n == 1 then
    n
  else
    fib(n - 1) + fib(n - 2)
}
\end{verbatim}
Kwoka supports higher order and polymorphic functions.
\begin{verbatim}
fn compose(f, g)
{
  \x => f(g(x))
}
\end{verbatim}
Something about algebraic effects
\begin{verbatim}
fn makeGreeting(name)
{
  Hello() ^ " " ^ name
}

effect Hello
{
  Hello() :: String
}

fn main()
{
  handle<Hello>(makeGreeting("General Kenobi."))
  {
    return(x) => (),
    Hello() => resume("Hello there!")
  }
}
\end{verbatim}
Popular example of usage of the algebraic effects is using them to
express exceptions.
\begin{verbatim}
effect Exc
{
  Raise(String)
}

fn saveDiv(n, m)
{
  if m == 0 then
    Raise("Division by zero")
  else
    n / m
}

fn main()
{
  handle<Exc>(saveDiv(4, 5))
  {
    return(x) => x,
    Raise(s) => 0
  }
}
\end{verbatim}
\section{Type Inference}
\section{Implementation}
\section{Installing and running Kwoka}
Kwoka requires {\href{https://docs.haskellstack.org/en/stable/README/}{\color{blue}stack}} for building.
\begin{itemize}
    \item Building: \verb+stack build+
    \item Running: \verb+stack run yourSourceFile.kwoka+
    \item Running unit tests: \verb+stack test+
\end{itemize}

\section{Further work}

\bibliographystyle{unsrt}
\bibliography{biblo.bib}

\end{document}